\documentclass{article}
\usepackage{listings}
\usepackage{amsmath}
\usepackage{color} %red, green, blue, yellow, cyan, magenta, black, white
\begin{document}
\begin{titlepage}
	\centering
	{\scshape\Large FMA240 - Handin 1\par}
	\vspace{1cm}
	Kristoffer Lundgrend \texttt{<kem01klu@student.lu.se>}\par
	Stefan Eng \texttt{<atn08sen@student.lu.se>}\par
    \vspace{1cm}
	\today\par
\end{titlepage}

\definecolor{mygreen}{RGB}{28,172,0} % color values Red, Green, Blue
\definecolor{mylilas}{RGB}{170,55,241}

\lstset{language=Matlab,%
    %basicstyle=\color{red},
    breaklines=false,%
    morekeywords={matlab2tikz},
    keywordstyle=\color{blue},%
    morekeywords=[2]{1}, keywordstyle=[2]{\color{black}},
    identifierstyle=\color{black},%
    stringstyle=\color{mylilas},
    commentstyle=\color{mygreen},%
    showstringspaces=false,%without this there will be a symbol in the places where there is a space
    emph=[1]{for,end,break},emphstyle=[1]\color{red}, %some words to emphasise
    %emph=[2]{word1,word2}, emphstyle=[2]{style},
}

\section*{Exercise 1}

    See attached file \textit{checkbasic1.m}.

\section*{Exercise 2}

    Run the function defined in exercise 1 for different permutation of
    basic variable selections.

    \begin{lstlisting}
        % Shared variables for all runs.
        A = [1 1 1 0 0; 1 0 0 1 0; 8 20 0 0 1];
        b = [1;3/4;10];
        c = [2; 1; 0; 0; 0];
    \end{lstlisting}

    Running with the third, forth and fifth variable as basic variables:

    \begin{lstlisting}
        basic = [3 4 5];
        [tab,x,bas,feas,opt] = checkbasic1(A,b,c,basic);
    \end{lstlisting}

    Results in the following tableau:

    \begin{center}
        \begin{tabular}{ | c | c c c c c | c | }
            \hline
                   & $x_1$ & $x_2$ & $x_3$ & $x_4$ & $x_5$ &  \\
            \hline
            $x_3$ & 1.00  & 1.00   & 1.00 & 0   &  0   & 1.00  \\
            $x_4$ & 1.00  & 0      & 0    & 1.00&  0   & 0.75 \\
            $x_5$ & 8.00  & 20.00  & 0    & 0   &  1.00& 10.00 \\
            \hline
                  & -2.00  & -1.00  & 0    & 0   &  0   & 0 \\
            \hline
        \end{tabular}
    \end{center}

    Running with the third, first and fifth variable as basic variables:

    \begin{lstlisting}
        basic = [3 1 5];
        [tab,x,bas,feas,opt] = checkbasic1(A,b,c,basic);
    \end{lstlisting}

    Results in the following tableau:

    \begin{center}
        \begin{tabular}{ | c | c c c c c | c | }
            \hline
                   & $x_1$ & $x_2$ & $x_3$ & $x_4$ & $x_5$ &  \\
            \hline
            $x_3$ & 0  & 1.00   & 1.00 & -1.00   &  0   & 0.25  \\
            $x_1$ & 1.00  & 0      & 0    & 1.00&  0   & 0.75 \\
            $x_5$ & 0  & -1.00  & 0    & 2.00  & 1.00 & 4.00 \\
            \hline
                  & 0  & -1.00  & 0    & 2.00   &  0   & 1.50 \\
            \hline
        \end{tabular}
    \end{center}

    And lastly running with the third, first and second variable selected as
    basic variables.

    \begin{lstlisting}
        basic = [3 1 2];
        [tab,x,bas,feas,opt] = checkbasic1(A,b,c,basic);
    \end{lstlisting}

    Resulting in the following tableau.

    \begin{center}
        \begin{tabular}{ | c | c c c c c | c | }
            \hline
                   & $x_1$ & $x_2$ & $x_3$ & $x_4$ & $x_5$ &  \\
            \hline
            $x_3$ & 0  & 0   & 1.00 & -0.60   &  0.05   & 0.05  \\
            $x_1$ & 1.00  & 0      & 0    & 1.00&  0   & 0.75 \\
            $x_2$ & 0  & 0  & 0    & 1.60  & 0.05 & 0.20 \\
            \hline
                  & 0  & 0  & 0    & 1.60   &  0.05  & 1.70 \\
            \hline
        \end{tabular}
    \end{center}

\section*{Exercise 3}

    For exercise 3, the checkbasic1 function was run with the following data:

    \begin{lstlisting}
        % Shared data.
        A = [3 2 1 0 0; 5 1 1 1 0; 2 5 1 0 1];
        b = [1; 3; 4];
        % Converting min-problem to max-problem.
        c = [-1; -1; -1; -1; -1];

        % First tableau.
        basicvars = [3 4 5];
        [tableau,~,~,~,~] = checkbasic1(A,b,c,basicvars);
        disp('Problem 3a');
        tableau

        % Second tableau.
        basicvars = [2 4 5];
        [tableau,~,~,~,~] = checkbasic1(A,b,c,basicvars);
        disp('Problem 3b');
        tableau
    \end{lstlisting}

    Generating the following tableau's:

        \begin{center}
            \begin{tabular}{ | c | c c c c c | c | }
                \hline
                       & $x_1$ & $x_2$ & $x_3$ & $x_4$ & $x_5$ &  \\
                \hline
                $x_3$ & 3  & 2   & 1 & 0   &  0   & 1  \\
                $x_4$ & 2  & -1 & 0    & 1 &  0   & 2 \\
                $x_5$ & -1  & 3  & 0    & 0  & 1 & 3 \\
                \hline
                      & -3  & -3  & 0    & 0   &  0  & -6 \\
                \hline
            \end{tabular}
        \end{center}

        \begin{center}
            \begin{tabular}{ | c | c c c c c | c | }
                \hline
                       & $x_1$ & $x_2$ & $x_3$ & $x_4$ & $x_5$ &  \\
                \hline
                $x_2$ & 1.50  & 1.00   & 0.50 & 0  &  0   & 0.50  \\
                $x_4$ & 3.50  & 0      & 0.50 & 1.00&  0   & 2.50 \\
                $x_5$ & -5.50  & 0  & -1.5    & 0  & 1.0 & 1.50 \\
                \hline
                      & 1.50  & 0  & 1.50    & 0   &  0  & -4.5 \\
                \hline
            \end{tabular}
        \end{center}

\section*{Exercise 4}

    Use the two-phase method to solve the following equation.

    \begin{center}
         max $ z = x_1 - 2x_2 - 3x_3 - x_4 - x_5 + 2x_6 $
    \end{center}

    subject to:

    \begin{center}
        \begin{math}
           \begin{cases}
                    x_1+2x_2+2x_3+x_4+x_5 = 12 \\
                    x_1+2x_2+x_3+x_4+2x_5+x_6 = 18 \\
                    3x_1+6x_2+2x_3+x_4+3x_5 = 24 \\
             \end{cases}
        \end{math}
    \end{center}

    Setting up the following data and using the simplex method

    \begin{lstlisting}
        A=[1 2 2 1 1 0 1 0 0; ...
           1 2 1 1 2 1 0 1 0; ...
           3 6 2 1 3 0 0 0 1];
        b=[12;18;24];
        c=[0;0;0;0;0;0;-1;-1;-1];
        basicvars=[7 8 9];
    \end{lstlisting}

    gives the following tableau:

        \begin{center}
            \begin{tabular}{ | c | c c c c c c c c c | c | }
                \hline
                       & $x_1$ & $x_2$ & $x_3$ & $x_4$ & $x_5$ & $x_6$ & $x_7$ & $x_8$ & $x_9$ &\\
                \hline
                  $x_3$ & 0& 0& 1.00& 0.50& 0& 0& 0.750& 0& -0.250& 3.00\\
                  $x_6$ & 0& 0& 0& 0.50& 1.00& 1.00& -0.25& 1.00& -0.25 & 9.00\\
                  $x_2$ & 0.50& 1.00& 0& 0& 0.50& 0& -0.25 & 0& 0.25 & 3.00\\
                \hline
                       & 0& 0& 0& 0& 0& 0& 1.00& 1.00& 1.00& 0 \\
                \hline
            \end{tabular}
        \end{center}

    base new data on second input and previous tableau:

    \begin{lstlisting}
        % New data.
        A = [1 2 2 1 1 0; ...
             1 2 1 1 2 1; ...
             3 6 2 1 3 0 ];
        c = [1; -2; -3; -1; -1; 2];
        basicvars = [3 6 2];
    \end{lstlisting}

    which, using the simplex method gives the following tableau:

    \begin{center}
        \begin{tabular}{ | c | c c c c c c | c | }
            \hline
                          & $x_1$ & $x_2$ & $x_3$ & $x_4$ & $x_5$ & $x_6$ &\\
            \hline
                   $x_3$  & 0& 0& 1.00& 0.50& 0& 0& 3.00 \\
                   $x_6$  & 0& 0& 0& 0.50& 1.00& 1.00& 9.00 \\
                   $x_1$  & 1.00& 2.00& 0& 0& 1.00& 0& 6.00 \\
            \hline
                          & 0& 4.00& 0& 0.50& 4.00& 0& 15.00 \\
            \hline
        \end{tabular}
    \end{center}

\section*{Exercise 5}
\section*{Exercise 6}
\section*{Exercise 7}


\end{document}
