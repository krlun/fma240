\documentclass{article}
\usepackage{listings}
\usepackage{amsmath}
\usepackage{tikz}
\usepackage{color} %red, green, blue, yellow, cyan, magenta, black, white
\begin{document}

\newcommand\encircle[1]{%
  \tikz[baseline=(X.base)]
    \node (X) [draw, shape=circle, inner sep=0] {\strut #1};}

\definecolor{mygreen}{RGB}{28,172,0} % color values Red, Green, Blue
\definecolor{mylilas}{RGB}{170,55,241}

\lstset{language=Matlab,%
    %basicstyle=\color{red},
    breaklines=false,%
    morekeywords={matlab2tikz},
    keywordstyle=\color{blue},%
    morekeywords=[2]{1}, keywordstyle=[2]{\color{black}},
    identifierstyle=\color{black},%
    stringstyle=\color{mylilas},
    commentstyle=\color{mygreen},%
    showstringspaces=false,%without this there will be a symbol in the places where there is a space
    emph=[1]{for,end,break},emphstyle=[1]\color{red}, %some words to emphasise
    %emph=[2]{word1,word2}, emphstyle=[2]{style},
}

\begin{centering}
	{\scshape\Large FMA240 - Handin 1\par}
	\vspace{0.5cm}
	Kristoffer Lundgren \texttt{<kem01klu@student.lu.se>}\par
	Stefan Eng \texttt{<atn08sen@student.lu.se>}\par
    \vspace{0.5cm}
	\today\par
    \rule{\textwidth}{0.4pt}
\end{centering}

\section*{Exercise 1}

    See attached file \textit{checkbasic1.m}.

\section*{Exercise 2}

    Run the function defined in exercise 1 for different permutation of
    basic variable selections.

    \begin{lstlisting}
        % Shared variables for all runs.
        A = [1 1 1 0 0; 1 0 0 1 0; 8 20 0 0 1];
        b = [1;3/4;10];
        c = [2; 1; 0; 0; 0];
    \end{lstlisting}

    Running with the third, forth and fifth variable as basic variables:

    \begin{lstlisting}
        basic = [3 4 5];
        [tab,x,bas,feas,opt] = checkbasic1(A,b,c,basic);
    \end{lstlisting}

    Results in the following tableau:

    \begin{center}
        \begin{tabular}{ | c | c c c c c | c | }
            \hline
                   & $x_1$ & $x_2$ & $x_3$ & $x_4$ & $x_5$ &  \\
            \hline
            $x_3$ & 1.00  & 1.00   & 1.00 & 0   &  0   & 1.00  \\
            $x_4$ & 1.00  & 0      & 0    & 1.00&  0   & 0.75 \\
            $x_5$ & 8.00  & 20.00  & 0    & 0   &  1.00& 10.00 \\
            \hline
                  & -2.00  & -1.00  & 0    & 0   &  0   & 0 \\
            \hline
        \end{tabular}
    \end{center}

    Running with the third, first and fifth variable as basic variables:

    \begin{lstlisting}
        basic = [3 1 5];
        [tab,x,bas,feas,opt] = checkbasic1(A,b,c,basic);
    \end{lstlisting}

    Results in the following tableau:

    \begin{center}
        \begin{tabular}{ | c | c c c c c | c | }
            \hline
                   & $x_1$ & $x_2$ & $x_3$ & $x_4$ & $x_5$ &  \\
            \hline
            $x_3$ & 0  & 1.00   & 1.00 & -1.00   &  0   & 0.25  \\
            $x_1$ & 1.00  & 0      & 0    & 1.00&  0   & 0.75 \\
            $x_5$ & 0  & -1.00  & 0    & 2.00  & 1.00 & 4.00 \\
            \hline
                  & 0  & -1.00  & 0    & 2.00   &  0   & 1.50 \\
            \hline
        \end{tabular}
    \end{center}

    And lastly running with the third, first and second variable selected as
    basic variables.

    \begin{lstlisting}
        basic = [3 1 2];
        [tab,x,bas,feas,opt] = checkbasic1(A,b,c,basic);
    \end{lstlisting}

    Resulting in the following tableau.

    \begin{center}
        \begin{tabular}{ | c | c c c c c | c | }
            \hline
                   & $x_1$ & $x_2$ & $x_3$ & $x_4$ & $x_5$ &  \\
            \hline
            $x_3$ & 0  & 0   & 1.00 & -0.60   &  0.05   & 0.05  \\
            $x_1$ & 1.00  & 0      & 0    & 1.00&  0   & 0.75 \\
            $x_2$ & 0  & 0  & 0    & 1.60  & 0.05 & 0.20 \\
            \hline
                  & 0  & 0  & 0    & 1.60   &  0.05  & 1.70 \\
            \hline
        \end{tabular}
    \end{center}

\section*{Exercise 3}

    For exercise 3, the checkbasic1 function was run with the following data:

    \begin{lstlisting}
        % Shared data.
        A = [3 2 1 0 0; 5 1 1 1 0; 2 5 1 0 1];
        b = [1; 3; 4];
        % Converting min-problem to max-problem.
        c = [-1; -1; -1; -1; -1];

        % First tableau.
        basicvars = [3 4 5];
        [tableau,~,~,~,~] = checkbasic1(A,b,c,basicvars);
        disp('Problem 3a');
        tableau

        % Second tableau.
        basicvars = [2 4 5];
        [tableau,~,~,~,~] = checkbasic1(A,b,c,basicvars);
        disp('Problem 3b');
        tableau
    \end{lstlisting}

    Generating the following tableau's:

        \begin{center}
            \begin{tabular}{ | c | c c c c c | c | }
                \hline
                      & $x_0$ & $x_2$ & $x_3$ & $x_4$ & $x_5$ &  \\
                \hline
                $x_3$ & 3  & 2   & 1 & 0   &  0   & 1  \\
                $x_4$ & 2  & -1 & 0    & 1 &  0   & 2 \\
                $x_5$ & -1  & 3  & 0    & 0  & 1 & 3 \\
                \hline
                      & -3  & -3  & 0    & 0   &  0  & -6 \\
                \hline
            \end{tabular}
        \end{center}

        \begin{center}
            \begin{tabular}{ | c | c c c c c | c | }
                \hline
                       & $x_1$ & $x_2$ & $x_3$ & $x_4$ & $x_5$ &  \\
                \hline
                $x_2$ & 1.50  & 1.00   & 0.50 & 0  &  0   & 0.50  \\
                $x_4$ & 3.50  & 0      & 0.50 & 1.00&  0   & 2.50 \\
                $x_5$ & -5.50  & 0  & -1.5    & 0  & 1.0 & 1.50 \\
                \hline
                      & 1.50  & 0  & 1.50    & 0   &  0  & -4.5 \\
                \hline
            \end{tabular}
        \end{center}

\section*{Exercise 4}

    Use the two-phase method to solve the following equation.

    \begin{center}
         max $ z = x_1 - 2x_2 - 3x_3 - x_4 - x_5 + 2x_6 $
    \end{center}

    subject to:

    \begin{center}
        \begin{math}
           \begin{cases}
                    x_1+2x_2+2x_3+x_4+x_5 = 12 \\
                    x_1+2x_2+x_3+x_4+2x_5+x_6 = 18 \\
                    3x_1+6x_2+2x_3+x_4+3x_5 = 24 \\
             \end{cases}
        \end{math}
    \end{center}

    Setting up the following data and using the simplex method

    \begin{lstlisting}
        A=[1 2 2 1 1 0 1 0 0; ...
           1 2 1 1 2 1 0 1 0; ...
           3 6 2 1 3 0 0 0 1];
        b=[12;18;24];
        c=[0;0;0;0;0;0;-1;-1;-1];
        basicvars=[7 8 9];
    \end{lstlisting}

    gives the following tableau:

        \begin{center}
            \begin{tabular}{ | c | c c c c c c c c c | c | }
                \hline
                       & $x_1$ & $x_2$ & $x_3$ & $x_4$ & $x_5$ & $x_6$ & $x_7$ & $x_8$ & $x_9$ &\\
                \hline
                  $x_3$ & 0& 0& 1.00& 0.50& 0& 0& 0.750& 0& -0.250& 3.00\\
                  $x_6$ & 0& 0& 0& 0.50& 1.00& 1.00& -0.25& 1.00& -0.25 & 9.00\\
                  $x_2$ & 0.50& 1.00& 0& 0& 0.50& 0& -0.25 & 0& 0.25 & 3.00\\
                \hline
                       & 0& 0& 0& 0& 0& 0& 1.00& 1.00& 1.00& 0 \\
                \hline
            \end{tabular}
        \end{center}

    base new data on second input and previous tableau:

    \begin{lstlisting}
        % New data.
        A = [1 2 2 1 1 0; ...
             1 2 1 1 2 1; ...
             3 6 2 1 3 0 ];
        c = [1; -2; -3; -1; -1; 2];
        basicvars = [3 6 2];
    \end{lstlisting}

    which, using the simplex method gives the following tableau:

    \begin{center}
        \begin{tabular}{ | c | c c c c c c | c | }
            \hline
                          & $x_1$ & $x_2$ & $x_3$ & $x_4$ & $x_5$ & $x_6$ &\\
            \hline
                   $x_3$  & 0& 0& 1.00& 0.50& 0& 0& 3.00 \\
                   $x_6$  & 0& 0& 0& 0.50& 1.00& 1.00& 9.00 \\
                   $x_1$  & 1.00& 2.00& 0& 0& 1.00& 0& 6.00 \\
            \hline
                          & 0& 4.00& 0& 0.50& 4.00& 0& 15.00 \\
            \hline
        \end{tabular}
    \end{center}

\section*{Exercise 5}

    Manually determine the optimal result of the following data on canonical
    form using the simplex method.

    \begin{lstlisting}
        A=[2,-3,2,1,0; ...
           -1,1,1,0,1];
        b=[3;5];
        c=[3;2;1;0;0];
    \end{lstlisting}

    or, represented differently:

    \begin{center}
        maximize $ z = c^Tx $
    \end{center}

    subject to:

    \begin{center}
        \begin{math}
            \begin{cases}
                2x_1 - 3x_2 + 2x_3 + x_4 = 3\\
                -x_1 - x_2 + x_3 + x_5 = 5
            \end{cases}
        \end{math}
    \end{center}

    Running the given data through the checkbasic1 function with $x_4$ and
    $x_5$ as basic variables results in the following tableau:

    \begin{center}
        \begin{tabular}{ c | c | c c c c c | c | }
            \multicolumn{1}{c}{}  & \multicolumn{1}{c}{}  &       &       &       & \multicolumn{1}{c}{}\\
            \cline{2-8}
            &  & $x_1$ & $x_2$ & $x_3$ & $x_4$ & $x_5$ &\\
            \cline{2-8}
                          &  $x_4$ &2 &-3 &2 &1 &0 &3\\
                          &  $x_5$ &-1 &1 &1 &0 &1 &5\\
            \cline{2-8}
                          &        &-3 &-2 &-1 &0 &0 &0\\
            \cline{2-8}
        \end{tabular}
    \end{center}

    Since -3 at $x_1$ is the most negative value in the object row $x_1$ is
    selected as the pivotal column. The pivot row is selected by calculating
    the $\theta$-ratios for $x_4$ and $x_5$ and taking the smallest. But since
    we don't use negative column values to calculate $\theta$-ratios, $x_4$ is
    the pivot row, which makes 2 the pivot.

    \begin{center}
        \begin{tabular}{ c | c | c c c c c | c | }
            \multicolumn{1}{c}{} & \multicolumn{1}{c}{} & \multicolumn{1}{c}{$\downarrow$}  &       &       &       & \multicolumn{1}{c}{}\\
            \cline{2-8}
            &  & $x_1$ & $x_2$ & $x_3$ & $x_4$ & $x_5$ &\\
            \cline{2-8}
            $\leftarrow$  &  $x_4$ &\encircle{2} &-3 &2 &1 &0 &3\\
                          &  $x_5$ &-1 &1 &1 &0 &1 &5\\
            \cline{2-8}
                          &        &-3 &-2 &-1 &0 &0 &0\\
            \cline{2-8}
        \end{tabular}
    \end{center}

    using checkbasic1 with the new basic values gives a new tableau. As
    earlier, by checking the most negative value in the object row together
    with the smallest $\theta$-ratio (1.5 vs. 3.25) $x_3$ is the ingoing
    variable and $x_1$ the outgoing.

    \begin{center}
        \begin{tabular}{ c | c | c c c c c | c | }
            \multicolumn{1}{c}{}  & \multicolumn{1}{c}{}  &    &       &  $\downarrow$     & \multicolumn{1}{c}{}\\
            \cline{2-8}
            &  & $x_1$ & $x_2$ & $x_3$ & $x_4$ & $x_5$ &\\
            \cline{2-8}
            $\leftarrow$  & $x_1$ &  1&  -1.5&   \encircle{1}&   0.5&        0&   1.5\\
                          & $x_5$ &       0&  -0.5&   2&   0.5&   1&   6.5\\
            \cline{2-8}
            &  &       0&   6.5&  -2&  -1.5&        0&  -4.5\\
            \cline{2-8}
        \end{tabular}
    \end{center}

    next iteration:

    \begin{center}
        \begin{tabular}{ c | c | c c c c c | c | }
            \multicolumn{1}{c}{}  & \multicolumn{1}{c}{}  &    & $\downarrow$ &     & \multicolumn{1}{c}{}\\
            \cline{2-8}
            &  & $x_1$ & $x_2$ & $x_3$ & $x_4$ & $x_5$ &\\
            \cline{2-8}
            & $x_3$ & 1& -1.5& 1& 0.5& 0& 3\\
            $\leftarrow$ & $x_5$ & -2& \encircle{2.5}& 0& -0.5& 1& -1\\
            \cline{2-8}
            & & -2& -3.5& 0& 0.5& 0& -6\\
            \cline{2-8}
        \end{tabular}
    \end{center}

    next iteration:

    \begin{center}
        \begin{tabular}{ c | c | c c c c c | c | }
            \multicolumn{1}{c}{}  & \multicolumn{1}{c}{}  &    & &     & \multicolumn{1}{c}{}\\
            \cline{2-8}
            &  & $x_1$ & $x_2$ & $x_3$ & $x_4$ & $x_5$ &\\
            \cline{2-8}
            & $x_3$ & -0.6667& 1& -0.6667& -0.3333& 0& -2\\
            & $x_2$ & -0.3333& 0& 1.6667& 0.3333& 1& 10\\
            \cline{2-8}
            & & 4.3333 & 0& 2.3333& 0.6667& 0& 7\\
            \cline{2-8}
        \end{tabular}
    \end{center}

    Since there are no more negative values present in the objective-row, no
    more iterations are possible. But since this iteration objective-row has 0
    in the columns labeled by the basic variables and all variables labeled by
    non-basic variables are non-negative, the solution is optimal.

\section*{Exercise 6}

    Determine the total number of basic solutions and how many of them are
    feasible given the following data:

    \begin{lstlisting}
        A=[1,2,1,0;1,2,0,1];
        b=[4;7];
        c=[1;-1;0;0];
        basicvars = [3 4];
    \end{lstlisting}

    \noindent
    This was calculated with the help of checkbasic1, code can be found in the
    attached file \textit{calc\_basic.m}.
    The following formula gives the total number of basic solutions:

    \begin{equation}
        \binom{s}{m} = \frac{s!}{m!(s - m)!} = \binom{s}{s-m}
    \end{equation}

    \noindent
    where $s$ is the number of equations in the system, and $m$ is the number
    of columns or variables. ($s - m$ is equal to the number of
    basic variables.) By iterating over all possible combination of basic
    variables and using the checkbasic1 function to see if they are feasible or
    not, the following data is acquired. \\

    \noindent
    Number of basic solutions: 6. \\
    Number of basic feasible solutions: 3 ([1 4], [2 4], [3 4]).

\section*{Exercise 7}

    Solve the following problem:

    \begin{lstlisting}
        A=[-1,1;-1,0;8,-20];
        b=[1;3/4;10];
        c=[2;1];
    \end{lstlisting}

    which correspond to the following equation system:

    \begin{center}
        \begin{math}
            \begin{cases}
                -x_1 + x_2 = 1 \\
                -x_1 = \frac{3}{4} \\
                8x_1 - 20x_2 = 10 \\
            \end{cases}
        \end{math}
    \end{center}

    First, the second equation states that $x_1 = -\frac{3}{4} < 0$ which violates the assumed
    constraint that $x_j \geq 0 , j=[1,2,\dots]$. Secondly, accepting that $x_1 =
    -\frac{3}{4}$; the first equation gives $x_2 = \frac{7}{4}$, which
    substituted together with $x_1$ into the third equation results in:
    $8(-\frac{3}{4}) - 20\frac{7}{4} = -41$, which conflicts with the value of
    10, stated in the system. This system is not possible to solve.

\end{document}
