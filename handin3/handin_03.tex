\documentclass{article}
\usepackage{listings}
\usepackage{amsmath}
\usepackage{tikz}
\usepackage{color} %red, green, blue, yellow, cyan, magenta, black, white

\begin{document}

\newcommand{\q}{\textbf{\huge{\textcolor{red}{?}}}}

\definecolor{mygreen}{RGB}{28,172,0} % color values Red, Green, Blue
\definecolor{mylilas}{RGB}{170,55,241}

\lstset{language=Matlab,%
    basicstyle=\footnotesize\ttfamily
    breaklines=false,%
    morekeywords={matlab2tikz},
    keywordstyle=\color{blue},%
    morekeywords=[2]{1}, keywordstyle=[2]{\color{black}},
    identifierstyle=\color{black},%
    stringstyle=\color{mylilas},
    commentstyle=\color{mygreen},%
    showstringspaces=false,%without this there will be a symbol in the places where there is a space
    emph=[1]{for,end,break},emphstyle=[1]\color{red}, %some words to emphasise
    %emph=[2]{word1,word2}, emphstyle=[2]{style},
}

\begin{centering}
	{\scshape\Large FMA240 - Handin 3\par}
	\vspace{0.5cm}
	Kristoffer Lundgren \texttt{<kem01klu@student.lu.se>}\par
	Stefan Eng \texttt{<atn08sen@student.lu.se>}\par
    \vspace{0.5cm}
	\today\par
    \rule{\textwidth}{0.4pt}
\end{centering}

\section*{Exercise 1}
  In a transportation problem we can always assume that the total supply equals
  the total demand. Even if this is not the case, a dummy column or a row with
  0 costs can be added to balance out the missing demand or supply.
  The general transportation problem can then be formulated as
  \begin{align*}
  & \text{Minimize } z = \sum_{i=1}^{m}\sum_{j=1}^n c_{ij}x_{ij} \\
  & \text{subject to}
    \label{eq1}
  \end{align*}
  \begin{align}
    \begin{cases}
      \sum_{j=1}^{n} x_{ij} = s_{i}, & i = 1, 2, ..., m \\
      \sum_{i=1}^{m} x_{ij} = d_{j}, & j = 1, 2, ..., n
    \end{cases}
    %\label{eq2}
  \end{align}
  \begin{equation}
    \sum_{i=1}^{m} s_{i} = \sum_{j=1}^n d_{j}
    \label{eq3}
  \end{equation}
  \begin{equation}
    x_{ij} \geq 0, i = 1, 2, ..., m, j = 1, 2, ..., n.
    \label{eq4}
  \end{equation}

  \noindent
  In the case of this handin, the cost matrix is given by
  \begin{equation}
    \mathbf{C} =
    \begin{pmatrix}
      11 & * & 8 & 8 \\
      7  & 5 & 6 & 12 \\
      7  & 6 & 8 & 5
    \end{pmatrix}
    \label{eq5}
  \end{equation}
  where $*$ signifies that supplier 1 cannot deliver to location 2. In
  addition, supplier 1 offers a discount for location 4: every unit after $20k$
  costs only 5, instead of 8.
  The supply vector is give by
  \begin{equation}
    \mathbf{s} =
    \begin{pmatrix}
      100k & 120k & 60k
    \end{pmatrix}^T
    \label{eq6}
  \end{equation}
  where $k = 10^3$. Finally, the demand vector is given by
  \begin{equation}
    \mathbf{d} =
    \begin{pmatrix}
      50k & 40k & 90k & 70k
    \end{pmatrix}^T
    \label{eq7}
  \end{equation}
  with $k = 10^3$ as above. It is obvious that the supply does not equal the
  demand in this case. In fact, there is $30k$ less demand than there is
  supply. As mentioned above, this can be taken care of by adding a dummy
  delivery location, which soaks up the remaining $30k$, with a unit
  transportation cost of $0$. This means that the cost matrix in equation
  (\ref{eq5}) becomes
  \begin{equation}
    \mathbf{C} =
    \begin{pmatrix}
      11 & * & 8 & 8  & 0\\
      7  & 5 & 6 & 12 & 0\\
      7  & 6 & 8 & 5  & 0
    \end{pmatrix}
    \label{eq8}
  \end{equation}
  as well as the demand vector in equation (\ref{eq7}) changing to
  \begin{equation}
    \mathbf{d} =
    \begin{pmatrix}
      50k & 40k & 90k & 70k & 30k
    \end{pmatrix}^T
    \label{eq9}
  \end{equation}

  \noindent
  To ensure that nothing gets delivered from supplier 1 to location 2 the
  constraints in equation (\ref{eq4}) must be modified in such a way that
  $x_{12} = 0$. The discount in turn can handled by dividing
  the problem in two cases: either location 4 buys 0 to 20k units from supplier
  1 or more than 20k units from supplier 1.
  To make the formulation of the mathematical optimization problem easier,
  location 4 is divided into two locations, 4a and 4b.
  In case 1, location 4b cannot buy from supplier 1, in
  case 2 location 4a cannot buy from supplier 2 and 3. The best solution for
  either of these two cases is then accepted as the answer to the original
  problem. This gives case 1 the following mathematical definition:
 % m = 3
 % n = 6
  \begin{align*}
  & \text{Minimize } z = \sum_{i=1}^{3}\sum_{j=1}^6 c_{ij}x_{ij} \\
  & \text{subject to}
    \label{eq10}
  \end{align*}
  \begin{align}
    \begin{cases}
      \sum_{j=1}^{6} x_{ij} = s_{i}, & i = 1, 2, 3 \\
      \sum_{i=1}^{3} x_{ij} = d_{j}, & j = 1, 2, ..., 6
    \end{cases}
    %\label{eq11}
  \end{align}
  \begin{equation}
    \sum_{i=1}^{3} s_{i} = \sum_{j=1}^6 d_{j}
    \label{eq12}
  \end{equation}
  \begin{equation}
    x_{ij} \geq 0, i = 1, 2, 3, j = 1, 2, ..., 6, x_{12} = 0 x_{15} = 0.
    \label{eq13}
  \end{equation}
  with
  \begin{equation}
    \mathbf{C} =
    \begin{pmatrix}
      11 & * & 8 & 8  & *  & 0\\
      7  & 5 & 6 & 12 & 12 & 0\\
      7  & 6 & 8 & 5  & 5  & 0
    \end{pmatrix}
    \label{eq14}
  \end{equation}
  \begin{equation}
    \mathbf{d} =
    \begin{pmatrix}
      50k & 40k & 90k & 20k & 50k & 70k & 30k
    \end{pmatrix}^T
    \label{eq15}
  \end{equation}
  and
  \begin{equation}
    \mathbf{s} =
    \begin{pmatrix}
      100k & 120k & 60k
    \end{pmatrix}^T
    \label{eq16}
  \end{equation}

  \noindent
  And for case 2:
   \begin{align*}
  & \text{Minimize } z = \sum_{i=1}^{3}\sum_{j=1}^6 c_{ij}x_{ij} \\
  & \text{subject to}
    \label{eq17}
  \end{align*}
  \begin{align}
    \begin{cases}
      \sum_{j=1}^{6} x_{ij} = s_{i}, & i = 1, 2, 3 \\
      \sum_{i=1}^{3} x_{ij} = d_{j}, & j = 1, 2, ..., 6
    \end{cases}
    %\label{eq18}
  \end{align}
  \begin{equation}
    \sum_{i=1}^{3} s_{i} = \sum_{j=1}^6 d_{j}
    \label{eq19}
  \end{equation}
  \begin{equation}
    x_{ij} \geq 0, i = 1, 2, 3, j = 1, 2, ..., 6, x_{12} = 0, x_{24} = 0, x_{34} = 0.
    \label{eq20}
  \end{equation}
  with
  \begin{equation}
    \mathbf{C} =
    \begin{pmatrix}
      11 & * & 8 & 8  & 5  & 0\\
      7  & 5 & 6 & *  & 12 & 0\\
      7  & 6 & 8 & *  & 5  & 0
    \end{pmatrix}
    \label{eq21}
  \end{equation}
  \begin{equation}
    \mathbf{d} =
    \begin{pmatrix}
      50k & 40k & 90k & 20k & 50k & 70k & 30k
    \end{pmatrix}^T
    \label{eq22}
  \end{equation}
  and
  \begin{equation}
    \mathbf{s} =
    \begin{pmatrix}
      100k & 120k & 60k
    \end{pmatrix}^T
    \label{eq23}
  \end{equation}

  \section*{Exercise 2}
  To solve the transportation problems formulated in exercise 1 above, it is
  easiest to use the transportation algorithm (although, writing out all the
  constraints and running for example simplex would be possible, but more work).

  \noindent
  The transportation algorithm starts by finding a feasible solution to the
  problem by some method, for example the \textit{Minimum cell cost}
  method. This is a greedy algorithm occupying\q the
  transport routes that are the cheapest for each row in the cost matrix and
  meeting the respective supply and demand, starting with the first row and
  then doing the same for each consecutive row. The rest of the routes are set
  to 0.  To find an improvement to this starting solution (or verifying that it
  is the optimal solution), the reduced costs is defined as
  \begin{equation}
    \hat{c}_{ij} = c_{ij} - v_i - w_j
    \label{eq24}
  \end{equation}
  where $\hat{c}_{ij}$ would appear as slack variables when solving the dual
  system, and $v_i$, $w_j$ are the variables for the dual system.
  Complementary slackness can now be used, which states that if a solution is
  optimal, the following must hold:
  \begin{equation}
    \hat{c}_{ij}x_{ij} = 0 \Rightarrow \text{if } x_{ij} \neq 0 \text{ then } \hat{c}_{ij} = 0
  \end{equation}
  Creating a table for the $\hat{c}_{ij}$:s, these must be 0 where $x_{ij}$ is
  not. For these positions, it must then hold that $c_{ij} = v_i + w_j$ (as can
  be seen from equation (\ref{eq24})). Solving this equation system, the rest
  of the reduced costs can be calculated. The reduced costs should then be
  interpreted as the change in the total cost that we are optimizing for if one
  more unit is transported along that route; thus, finding a negative reduced
  cost means that the solution to the transport problem can be improved.
  Repeating this procedure until no more negative reduced costs are found, the
  optimal solution to the problem has then been found and can be returned as
  the answer.

  \section*{Exercise 3}
  Solving the two problems listed under the section ``Exercise 1'' above, the
  respective solutions are:

  \subsection*{Case 1}
  \begin{lstlisting}
>> ca = [11    100     8   8       100     0;
         7     5       6   12      12      0;
         7     6       8   5       5       0];
>> da = [50;   40;     90; 20;     50;     30];
>> sa = [100; 120; 60];
>> [xa, costa] = transport(sa, da, ca)
Optimal flow found:

xa =
     0     0    60    10     0    30
    50    40    30     0     0     0
     0     0     0    10    50     0


costa =
        1590
  \end{lstlisting}

  \subsection*{Case 2}
  \begin{lstlisting}
>> cb = [11    100     8   8       5       0;
         7     5       6   100     12      0;
         7     6       8   100     5       0];
>> db = [50;   40;     90; 20;     50;     30];
>> sb = [100; 120; 60];
>> [xb, costb] = transport(sb, db, cb)
Optimal flow found:

xb =
     0     0     0    20    50    30
     0    30    90     0     0     0
    50    10     0     0     0     0

costb =
        1510
  \end{lstlisting}

  \noindent
  The total cost is thus lower in the case where location 4 buys more than
  $20k$ units from supplier 1 and thereby obtaining the discount. In summary,
  location 1 should by $50k$ units from supplier 3, location 2 should by $30k$
  units from supplier 2 and $10k$ units from supplier 3, location 3 should by
  $90k$ units from supplier 2 and location 4 should by $70k$ units from
  supplier 1, in order to minimize the total costs. The total cost is 1510000.

\end{document}
